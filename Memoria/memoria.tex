\documentclass[10pt,a4paper]{article}
\usepackage{fontspec}
\usepackage{amsmath}
\usepackage{amsfonts}
\usepackage{amssymb}
\usepackage{graphicx}
\usepackage[spanish,es-nodecimaldot,es-lcroman,es-tabla,es-noshorthands]{babel}
\usepackage[left=3cm,right=2cm, bottom=4cm]{geometry}
\usepackage{subfigure}
\usepackage{longtable}
\usepackage{booktabs}
\usepackage{array}
\usepackage{color}

\newcolumntype{C}[1]{>{\centering\arraybackslash}m{#1}}

\definecolor{PUTCOLOR}{RGB}{46,74,255}
\definecolor{POSTCOLOR}{RGB}{187,136,0}
\definecolor{GETCOLOR}{RGB}{76,158,64}
\definecolor{DELETECOLOR}{RGB}{181,32,0}

\newcommand{\GET}{\colorbox{GETCOLOR}{GET}}
\newcommand{\POST}{\colorbox{POSTCOLOR}{POST}}
\newcommand{\PUT}{\colorbox{PUTCOLOR}{PUT}}
\newcommand{\DELETE}{\colorbox{DELETECOLOR}{DELETE}}

\setsansfont[Ligatures=TeX]{texgyreadventor}
\setmainfont[Ligatures=TeX]{texgyrepagella}

\input{portada}

\newcommand*{\autores}{
\begin{tabular}{r l}
GII+GIS: & Germán Alonso Azcutia \\
GIS:		 & Carlos Vázquez Sánchez \\
GIS+MAT: & José Ignacio Escribano Pablos
\end{tabular}
}

\begin{document}

\pagenumbering{alph}
\setcounter{page}{1}

\portada{Práctica 3}{Diseño de Aplicaciones Web}{Diseño e implementación de una  aplicación web \\ de rutas en AngularJS y Spring MVC}{\autores}{Móstoles}

\tableofcontents
\thispagestyle{empty}
\newpage

\pagenumbering{arabic}
\setcounter{page}{1}

\section{Introducción}

Para esta práctica, hemos decidido diseñar e implementar una aplicación que permita a sus usuarios ver, añadir rutas, de distintos tipos como senderismo, a caballo, a vela, etc. La aplicación es similar a Wikiloc.

\textcolor{red}{Añadir que tenemos aplicación de móvil en Google Play, y la API REST está desplegada en AWS}

\section{Arquitectura del sistema}

\section{Diagramas UML}

\subsection{Diagramas UML del backend}

\subsection{Diagramas UML del frontend}

\section{Recorrido por la aplicación}

\subsection{Aplicación de escritorio}

\subsection{Aplicación móvil}

\section{Conclusiones}

\appendix
\section{Diseño de la API REST}

\begin{longtable}[c]{@{}cllC{6cm}@{}}
\toprule
ID & URL & Método & Descripción\tabularnewline
\midrule
\endhead
 1 & /users & \GET & Muestra  todos los
usuarios\tabularnewline
\hline
 2 & /users/:id & \GET & Muestra información del
usuario :id\tabularnewline \hline
 3 & /users/:id/routes & \GET & Muestra todas las
rutas del usuario :id\tabularnewline \hline
 4 & /users/:id/friends & \GET & Muestra los amigos
del usuario :id\tabularnewline \hline
 5 & /users/:id/friends/routes & \GET & Muestra
todas las rutas de los amigos del usuario :id\tabularnewline \hline
 6 & /users/:id/routes/:id & \PUT & Actualiza la
ruta :id del usuario :id\tabularnewline \hline
 7 & /users/:id/ & \DELETE & Borra el usuario
:id\tabularnewline \hline
 8 & /users/:id/routes/:id & \DELETE & Borra la
ruta :id del usuario :id\tabularnewline \hline
 9 & /users/:id & \PUT & Actualiza el usuario
:id\tabularnewline \hline
 10 & /routes & \GET & Muestra todas las
rutas\tabularnewline \hline
 11 & /users/ & \POST & Añade un nuevo
usuario\tabularnewline \hline
 12 & /users/:id/routes & \POST & Añade una nueva
ruta al usuario :id\tabularnewline \hline
 13 & /users/:id/comments & \GET & Muestra todos
los comentarios del usuario :id\tabularnewline \hline
 14 & /users/:id/routes/:id/comments & \GET &
Muestra todos los comentarios de la ruta :id del usuario
:id\tabularnewline \hline
 15 & /users/:id/routes/:id/comments & \POST &
Añade un nuevo comentario a la ruta :id del usuario :id\tabularnewline \hline
 16 & /users/:id/routes/:id/comments/:id & \DELETE
& Borra el comentario :id del usuario :id en la ruta :id\tabularnewline \hline
 17 & /users/:id/routes/:id/comments/:id & \PUT &
Modifica el comentario :id del usuario :id en la ruta :id\tabularnewline \hline
 18 & /users/:id/routes/:id & \GET & Muestra
información de la ruta :id del usuario :id\tabularnewline \hline
 19 & /users/:id/friends & \POST & Añade un nuevo
amigo al usuario :id\tabularnewline
\bottomrule
\end{longtable}

\end{document}
